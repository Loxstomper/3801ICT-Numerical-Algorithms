\documentclass{article}
\begin{document}
		\begin{center}
			\vspace*{1cm}
			
			\Huge
			\textbf{3801ICT - Numerical Algorithms}
			
			\vspace{0.5cm}
			\LARGE
			Milestone One
			
			\vspace{1.5cm}
			
			\LARGE
			\textbf{Lochie Ashcroft - s5080439}
			
			\vfill
			\LARGE
			22/03/2019
			
		\end{center}
		\pagebreak
		
		Question 1
		
		\pagebreak
		
		\LARGE
		Question 2
		
		\normalsize
		When an aircraft is being tracked by radar then the position of the aircraft is determined by distance  (calculated  from  the  return  time  of  pulse)  and  the  sweep angle of the  radar.  To give  a  meaningful  radar  display  this  information  needs  to  be  converted  to  cartesian coordinates  and  velocity  and  acceleration  (both  vectors)  need  to  be  calculated.  Write  a program to perform this operation (use centered finite differences(second-ordercorrect)) and test your program with the data shown in the table below.
		
		\begin{table}[]
			\begin{tabular}{lllllll}
				T,s    & 200  & 202  & 204  & 206  & 208  & 210  \\
				θ, rad & 0.75 & 0.72 & 0.70 & 0.68 & 0.67 & 0.66 \\
				R, m   & 5120 & 5370 & 5560 & 5800 & 6030 & 6240
			\end{tabular}
		\end{table}
		
		\pagebreak
		
		Question 3

\end{document}